\subsection{Feature engineering}

Generalmente, el conjunto de datos a estudiar contiene más información de la que puede utilizar un modelo, entre estos datos se encuentran variables categóricas, o numéricas que, por si solas, no aportan información estadísticamente relevante. Por estos motivos, existe una técnica llamada \fe, que busca reutilizar esta información, combinándola de diversas maneras (ponderándolas, aplicándoles operadores lógicos, etc.), y así derivar una nueva característica que sí exhiba una relación relevante. Estas nuevas características pueden ser utilizadas para segmentar el conjunto de datos, con la finalidad de aumentar la precisión del modelo.

