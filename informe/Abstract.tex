\section{Abstract}

En este trabajo se estudia el método de \emph{Least squares} (o cuadrados mínimos lineales) aplicado a la predicción de características de inmuebles del mercado mexicano. Proponemos distintos modelos de regresiones, como pueden ser lineales o polinomiales; introducimos el concepto de segmentación del conjunto de datos, por características como el tipo de propiedad, o la antiguedad de la propiedad; y también construimos nuevas características a partir de las que ya presentes (\textit{feature engineering}).

Concluimos que para su simpleza, los regresores empleados tienen buenos resultados medidos con distintas métricas, por ejemplo r2, para la predicción de precio (r2 = 0.54) y en especial metros cubiertos (r2 = 0.70) de un inmueble.

\keywords{least squares, cml, statistical analysis, predictive models, cross validation, feature engineering}
